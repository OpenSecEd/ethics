% $Id$
\documentclass[a4paper,nocourse]{miunasgn}
\usepackage[utf8]{inputenc}
\usepackage[english,swedish]{babel}
\usepackage[hyphens]{url}
\usepackage{hyperref}
\usepackage{prettyref,varioref}
\usepackage{natbib}
\usepackage{listings}
\usepackage[today,nofancy]{svninfo}
\usepackage[natbib,varioref,prettyref,listings]{miunmisc}

\svnInfo $Id$

%\printanswers

\courseid{DT130G}
\course{Specialiserad applikationsutveckling för Android}
\assignmenttype{Seminarium}
\title{Säkerhet}
\author{Daniel Bosk\footnote{%
	Detta verk är tillgängliggjort under licensen Creative Commons 
	Erkännande-DelaLika 2.5 Sverige (CC BY-SA 2.5 SE).
	För att se en sammanfattning och kopia av licenstexten besök URL 
	\url{http://creativecommons.org/licenses/by-sa/2.5/se/}.
}}
\date{\svnId}

\begin{document}
\maketitle
\thispagestyle{foot}
\tableofcontents


\section{Introduktion}
\label{sec:Introduktion}
\noindent
Då vår värld blir alltmer uppkopplad och information blir enklare att kopiera 
ökar även behovet att skydda informationen.
Informationen varierar mellan allt från semesterbilder till sekretessbelagd 
data.

Ett exempel som illustrerar behovet av att stödja användaren för att skydda sig 
ges av \citet{Roberts2012wia}, där information om den norska kungafamiljens 
geografiska position delades nästintill i realtid -- information som av 
säkerhetsskäl normalt hålls hemlig.
Information läckte ut genom att ett av barnen postade geotaggade bilder direkt 
till Instagram.

Samma säkerhetslucka kan ha stora konsekvenser även för vanliga människor, 
trots att deras position normalt inte är sekretessbelagd, då det kan användas 
exempelvis för att veta när huset är mest tillgängligt för inbrott.
Ett proof-of-concept är WeKnowYourHouse.com \citep{Brading2012tpl}.


\section{Syfte}
\label{sec:Syfte}
\noindent
Detta seminarium handlar om att diskutera hur mycket ansvar utvecklare av olika 
tjänster har för att skydda användare och vilka metoder som finns tillgängliga.
Det syftar också till att ta upp till diskussion vilken information som 
utvecklare får samla in genom användningen av en tjänst och vad som krävs för 
att få samla in och lagra olika typer av information.

Det övergripande syftet med detta seminarium sammanfattas i följande två 
punkter:
\begin{itemize}
	\item Value and argue about different ethical aspects of computer security, 
e.g.\ possibilities for surveillance, and its consequences in society.

\end{itemize}


\section{Läsanvisningar}
\label{sec:Lasanvisningar}
\noindent
% $Id$
För att delta på seminariet krävs att du läst följande material:
\begin{itemize}
	\item \emph{Smartphones: Information security risks, opportunities and 
		recommendations for users} \cite{enisa2010sis},
	\item \emph{Smartphone Secure Development Guidelines for App Developers} 
		\cite{enisa2011ssd},
	\item \emph{Privacy considerations of online behavioural tracking} 
		\cite{enisa2012pco}, och
	\item \emph{Privacy Design Guidelines for Mobile Application Development} 
		\cite{GSMA2012pdg},
	\item \emph{OWASP Top 10 -- 2010} \cite{owasp2010ott}.
\end{itemize}



\section{Genomförande}
\noindent
Läs igenom materialet och anteckna dina tankar under läsningen.
Följande frågor kan verka som inspiration för dina funderingar kring 
materialet:
\begin{itemize}
	\item Vilken information tycker du att applikationer etc. får samla in?
	\item Hur ska informationen lagras?
	\item Vem får använda informationen, och till vad?
	\item Vilket stöd skulle du vilja ha från applikationer etc. för att kunna 
		skydda dig?
\end{itemize}


\section{Examination}
\label{sec:Examination}
\noindent
\emph{Aktivt} deltagande i seminariet krävs för godkänt betyg.


\bibliography{literature}
\end{document}
