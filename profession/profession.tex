\documentclass[a4paper]{article}
\usepackage[utf8]{inputenc}
\usepackage[swedish,english]{babel}
\usepackage[hyphens]{url}
\usepackage{hyperref}
\usepackage{cleveref}
\usepackage{csquotes}
\usepackage[natbib,style=numeric-comp,maxbibnames=99]{biblatex}
\addbibresource{profession.bib}

\usepackage[noamsthm]{beamerarticle}

\title{Seminar: The Computer Engineer's Code of Ethics}
\author{%
  Daniel Bosk
}
\institute{%
  Department of Information and Communication Systems\\
  Mid Sweden University, Sundsvall
}

\begin{document}
\maketitle


\section{Introduction}

Considering that our world becomes increasingly dependent on computer systems, 
it is important that those of us who are capable of controlling the computer 
systems do this in a responsible way.
But what is a responsible way?
This is the topic of ethics, the reasoning about moral obligations.
There are a great many texts written on the subject, we will in this assignment 
study some of the lighter ones.

As an example, since our digital world allows the copying of information more 
easily, more people do this.
This can happen in the form of people using a service for publishing 
photographs to friends, or to the entire world for that matter.
Sometimes the consequences of this can be severe.
For instance, when a child posted vacation pictures to 
Instagram~\cite{Roberts2012wia}.
This happened to be not any child, it was the child in the Norwegian royal 
family.
These images posted, did due to our technical development, contain the exact 
GPS coordinates of the location where the picture was taken.
The result being that the position of the royal family was published in almost 
real-time, information which is normally strictly secret.

This is also an issue for non-celebrities.
And the question is, where ends the responsibility of the engineers who 
contributed to these systems?
Should they be morally obliged to ensure the user can make an informed choice?
Or is that totally the responsibility of the user?
What if the design does not let the user make an informed choice?

We can then step it up a notch.
In app stores, e.g.\ Google Play, there are occasionally apps found which 
performs some morally questionable activities.
An Android app, ``Brightest Flashlight'' by Goldenshores Technologies, which 
could be found for free in the Play store, gathered data on users' locations 
and device identifiers, which the company later sold to 
advertisers~\cite{garber2014roundup}.

Another, more draconian, example concerns manufacturers of surveillance 
equipment, selling this equipment to authoritarian regimes for purposes of 
surveillance, censoring and in general repressing the people, as is the case 
for the companies Narus, BlueCoat Systems, Trovicor and 
Cisco~\cite{effsurveil}.
Narus sold equipment used for surveillance to the Egyptian government.
BlueCoat's equipment was used in Syria.
Germany-based Trovicor sold their technology to Bahrain; 
\blockcquote{effsurveil}{dozens of activists were tortured before and after 
  being shown transcripts of their text messages and phone conversations 
  captured from this technology}.

Is this type of engineering anywhere near ethically defensible?


\section{Aims}

This seminar aims to discuss the ethical issues regarding the problems 
described above, i.e.\ the moral responsibility of engineers.
After doing this assignment your should be able to:
\begin{itemize}
	\item Value and argue about different ethical aspects of computer security, 
e.g.\ possibilities for surveillance, and its consequences in society.

\end{itemize}


\section{Reading}
\label{sec:Reading}
% $Id$
För att delta på seminariet krävs att du läst följande material:
\begin{itemize}
	\item \emph{Smartphones: Information security risks, opportunities and 
		recommendations for users} \cite{enisa2010sis},
	\item \emph{Smartphone Secure Development Guidelines for App Developers} 
		\cite{enisa2011ssd},
	\item \emph{Privacy considerations of online behavioural tracking} 
		\cite{enisa2012pco}, och
	\item \emph{Privacy Design Guidelines for Mobile Application Development} 
		\cite{GSMA2012pdg},
	\item \emph{OWASP Top 10 -- 2010} \cite{owasp2010ott}.
\end{itemize}



\section{Assignment}
\label{sec:Tasks}
The first thing you should do is to analyse the cases presented in the texts 
you have read, i.e.\ the NSA's mass surveillance and FBI's wish for Apple to 
weaken the security of its products for the FBI\@.
You should analyse them from the perspective of the codes of ethics that you 
have read: which parts are \enquote{right} and which are \enquote{wrong}?
Are there conflicts?
Make sure to write down your conclusions and motivations.

You are now going to position yourself as one of the engineers working for the 
NSA or Apple.
Prepare a document with two parts.
The first part should contain arguments for why you should develop these 
exploits or intentionally weaken security, the second should contain arguments 
against developing these.
Your arguments should have their base in the codes of ethics you have read 
(include references), but also your own ethical values may be used in your 
argumentation --- but make sure to note when they conflict!

During the seminar, discuss your analyses and your positions.


\section{Examination}
\label{sec:Examination}
This assignment is examined through active participation in a seminar and 
a hand-in.
To prepare for this seminar, follow the instructions in \cref{sec:Tasks}.
Hand in the resulting document (analysis of the cases, your position) in the 
course platform and also bring them to the seminar.
Then you attend the seminar, see the schedule.
You must participate actively to pass this assignment.


\printbibliography{}
\end{document}
